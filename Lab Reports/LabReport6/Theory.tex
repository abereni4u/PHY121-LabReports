\section{Theory}
\vspace{-0.5cm}
\singlespacing

According to Newton's Second Law, \textit{force} is equal to \textit{mass} multiplied by \textit{acceleration}. \par

\vspace{-0.5cm}
\begin{equation*}
F = ma	
\end{equation*}

To increase the force of an object, either its mass or acceleration must increase. Since acceleration is inversely proportional to mass, increasing an object's mass decreases its acceleration and vice versa. In this lab, we varied the net force acting to accelerate a system, by shifting the distribution of mass within that system towards the hanging mass and keeping the total mass of the system constant. The only force acting vertically on the hanging mass is gravity, so by increasing its mass, its force will also increase.\par 

\textit{Friction} is the force keeping the glider in place. More specifically, when objects are at rest, \textit{static friction} acts on the object. Static friction is proportional to an object's \textit{normal force}, the opposing contact force to gravity as described by Newton's third law which states that every force has an equal and opposite force acting against it. Reducing an object's contact with a surface reduces the force of friction. The air track reduces the force of friction acting on the glider by blowing air against it. This allows the glider to move, as the force of the hanging mass falling is applied to it. The expected acceleration of the glider along the track can be calculated using the following equation: 

\vspace{-0.5cm}
{\centering
	\begin{equation*}
		a = \frac{F}{m}	
	\end{equation*}
\vspace{-0.5cm}
\begin{align*}
	\boldsymbol{F} &: \text{force of the hanging mass falling down.} \\
\boldsymbol{m} &: \text{mass of the entire system}
\end{align*}
}
 
Acceleration is the change in velocity over time. By making sure the glider has an initial velocity as close to zero as possible and measuring the time the glider travelled between photogates, we can calculate the acceleration of the glider using the following equation. 

\begin{equation*}
	a = \frac{2\Delta{x}}{t^2}	
\end{equation*}

\newpage

