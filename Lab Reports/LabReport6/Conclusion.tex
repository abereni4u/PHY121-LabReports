
\section{Conclusion}

\vspace{-0.5cm}
\singlespacing
The purpose of this lab was to observe the relationship between force and acceleration as described by Newton's Second Law. By systematically shifting the mass in our system, we varied the net force applied to it and increase the acceleration of the glider. The standard deviation of our trails were all less than 4\%, meaning our measurements were precise (see Table~\ref{tab:dataTabNew}). The percent error of our measured accelerations were relatively low to start, with increases towards the end of our trials, and our highest error percentage at 10.88\% (see Table~\ref{tab:dataTabNew}). A visual presentation of the gradual departure from expected values can be seen in Figure~\ref{fig:GFvA}, where our slope (mass) and y-intercept (static friction) are higher than expected. The mass of the system was measured to be 0.3049 kg (Table~\ref{tab:knownsTab}), as reflected by the slope of our expected accelerations in Figure~\ref{fig:GFvA}. The line created based off our measured accelerations resulted in a slope of 0.308 kg, which is 10.17\% less than our measured value for the mass of the system (see Figure~\ref{fig:GFvA}).\par 
 The source of these discrepancies is the manner in which our trials were conducted near the end. For all our trials, our measured accelerations were greater than our expected values (see Table~\ref{tab:dataTabNew}). This is most likely due to giving the glider a head start. Due to a mistake in data collection prior to these trials in which we didn't wait for the air track to fully startup before releasing the glider (resulting in much lower accelerations and a much higher mass than expected), we had less time to collect new data with better methods (PLEASE see Table~\ref{tab:dataTabOld} and Figure~\ref{fig:GFvA}). While the initial values of our new trials are much closer to expected, they deviate more as our time runs out. This explains why the slope of our measurements is lower than expected, as, based on Newton's Second Law, increasing acceleration produces a lower mass. Newton's Second Law also explains why the slope of our old trials are much higher than expected, since a lower acceleration would produce a greater mass (see Table~\ref{tab:dataTabOld}). Were the experiment to be conducted again, I'd simply wait for the air track to startup before letting go of the glider so that friction doesn't impact our results. 
