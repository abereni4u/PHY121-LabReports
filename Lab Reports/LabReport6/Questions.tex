
 \section{Questions}

\vspace{-0.5cm}
\singlespacing

\begin{enumerate}
	\item \textbf{Calculate the fractional discrepancy between the slope of your graph and the mass of your system as measured on the balance. Report this as $\frac{\Delta{M}}{M}$\%.}

		The fractional discrepancy between the slope of our graph \& the mass of the system as measured on the balance was 10.17\%. 	

	\item \textbf{Two possible sources of systematic error in this experiment are (a) giving the glider a head start, that is, starting up higher than x1 or accidentally giving it a push, and (b) if the air is not turned on high enough and there is friction. Discuss how each of these would affect your result (that is, the mass of the system as determined from the graph, and why. Accordingly, which do you think may have had a greater influence on your result?}

		Two possible sources of systematic error in this experiment are giving the slider a head start and not turning the air high enough to reduce friction on the air track. Giving the glider a head start would give it more time to accelerate, so our readings would be higher than they should be. Since mass is equal to $\frac{F}{a}$, where \textit{F} is force and \textit{a} is acceleration, an increased acceleration would result in a reduced mass. Not turning the air to a high enough setting to effectively reduce the force of friction and allow the glider to move, would result in the mass being higher than it should be since the measured acceleration would decrease. Based on our data, it's most likely the case that we gave the glider a head start during our trials, especially towards the end, as the mass after calculation is 10.17\% lower than our measured value of 0.3409 kg. \par

\item \textbf{What is the y-intercept of your graph? Is this what you expected? If not, what did you expect and why? What is the significance of the y-intercept you actually obtained?}

The y-intercept of our graph was 0.0058 N. We expected it to be close to zero so the results are within our expectations. The y-intercept represents the force needed to overcome static friction.

\item \textbf{During the experiment does the glider accelerate at a rate of $9.8 m/s^2$, slightly less, or significantly less along the frictionless track?}

During the experiment, the glider accelerates at a rate significantly less than 9.80 m/s$^2$.

\end{enumerate}

\newpage
