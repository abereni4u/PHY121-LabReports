
 \section{Questions}

\vspace{-0.5cm}
\singlespacing

\begin{enumerate}
	\item \textbf{What do you expect the acceleration of the picket fence to be? Is it exactly this value? If yours is higher or lower provide explanations as to why that might be.}

		We expected the acceleration of the picket fence to be around $9.80\thinspace{m/s^2}$. Our calculations ended up being slightly slower but well within 10.0\% of difference. The difference in values could be due to air resistance, or the small breeze produced by multiple bodies moving at the same time in a room.

	\item \textbf{Do you expect that the detector's measurement of time is exactly correct? What might affect the detector's measurement of time?}

	The detector itself has a few seconds of delay so the measurement of time wouldn't be completely accurate. It's also possible that the detection algorithm of the device is finely tuned such that the minute blending of colors at the borders on the picket fence would cause times to be off. The angle at which the picket fence was dropped could also have affected the Photogate's readings.

\item \textbf{What is your average value, standard deviation, and relative error for the acceleration due to gravity, \textit{g}?}

		For the acceleration due to gravity, \textit{g}, our average value was \textbf{$9.747\thinspace{m/s^2}$}, our standard deviation was \textbf{$0.06439\thinspace{m/s^2}$}, and the relative error was .5408\%.

\item \textbf{Would dropping the Picket Fence from higher above the Photogate timer change the measured value of g? Explain.}

	Dropping the Picket Fence from higher above the Photogate timer should not change the measured value of g by any significant amount. The measured value of g should change incredibly slightly since it's being dropped from a distance that is slightly farther from the Earth's surface.

\item \textbf{Does the acceleration found from the velocity vs time graphs in section 2.2 and 2.3 match each other? Should they be similar? Explain.}

	The acceleration found from the velocity vs time graphs in section 2.2 and 2.3 \textit{do} match other. The accelerations should be similar since no part of the track has been changed (angle and height is the same). The biggest source of difference between the graphs should be because of the cart being pushed up to the top of the track in 2.3, as opposed to being let go from it in 2.2. From the point where the cart starts coming back down in 2.3, the acceleration should be similar to 2.2, which the data shows. 

\item \textbf{Using $a=gsin\theta$ and $9.8\,m/s^2$ as the accepted value of g, calculate what you should have gotten from the motion detector for the acceleration along the ramp. Compare this value with the average value you found for the cart moving down the ramp and the average from the up and down the ramp. Is it within 10\% error?}

	After our calculations, the value that we should have gotten from the motion detector for the average acceleration along the ramp was -0.6105\,$m/s^2$. While our average for starting from the top of the track had a relative error of 12.9\%, our average for starting from the bottom had a relative error of 7.567\%. 
	
\end{enumerate}

