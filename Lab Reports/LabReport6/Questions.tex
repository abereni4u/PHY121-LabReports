
 \section{Questions}

\vspace{-0.5cm}
\singlespacing

\begin{enumerate}
	\item \textbf{Calculate the fractional discrepancy between the slope of your graph and the mass of your system as measured on the balance. Report this as $\frac{\Delta{M}}{M}$\%.}

		We expected the acceleration of the picket fence to be around $9.80\thinspace{m/s^2}$. Our calculations ended up being slightly slower but well within 10.0\% of difference. The difference in values could be due to air resistance, or the small breeze produced by multiple bodies moving at the same time in a room.

	\item \textbf{Two possible soruces of systematic error in this experiment are (a) giving the glider a head start, that is, starting up higher than x1 or accidentally giving it a push, and (b) if the air is not turned on high enough and there is friction. Discuss how each of these would affect your result (that is, the mass of the system as determined from the graph, and why. Accordingly, which do you think may have had a greater influence on your result?}

	The detector itself has a few seconds of delay so the measurement of time wouldn't be completely accurate. It's also possible that the detection algorithm of the device is finely tuned such that the minute blending of colors at the borders on the picket fence would cause times to be off. The angle at which the picket fence was dropped could also have affected the Photogate's readings.

\item \textbf{What is the y-intercept of your graph? Is this what you expected? If not, what did you expect and why? What is the significance of the y-intercept you actually obtained?}

		For the acceleration due to gravity, \textit{g}, our average value was \textbf{$9.747\thinspace{m/s^2}$}, our standard deviation was \textbf{$0.06439\thinspace{m/s^2}$}, and the relative error was .5408\%.

\item \textbf{During the experiment does the glider accelerate at a rate of $9.8 m/s^2$, slightly less, or significantly less along the frictionles track?}

	Dropping the Picket Fence from higher above the Photogate timer should not change the measured value of g by any significant amount. The measured value of g should change incredibly slightly since it's being dropped from a distance that is slightly farther from the Earth's surface.

\end{enumerate}


