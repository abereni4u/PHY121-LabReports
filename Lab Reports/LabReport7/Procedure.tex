
\section{Procedure}
\vspace{-0.5cm}
\singlespacing


%Free Fall

\subsection{Part 1: Starting Friction}

	\begin{enumerate}
		\item Used cleaning solution to wipe experiment surface of fingerprints.

		\item Measured mass of \textbf{wooden box }using scale.

		\item Connected \textbf{force sensor }to \textbf{Logger Pro}.

		\item Set range switch on \textbf{force sensor} to 50 N.

		\item Calibrated \textbf{force sensor }in \textbf{Logger Pro} by hanging 500
			g of weight (4.9 N) on it.

		\item Tied \textbf{wooden box} to \textbf{force sensor }and practiced
			pulling it with 1 kg of mass in the box.

		\item Repeated practicing until comfortable and graph output in \textbf{Logger
			Pro} was acceptable.
	\end{enumerate}	

\subsection{Part 2: Peak Static and Kinetic Friction}

	\begin{enumerate}[resume]
		\item Pulled \textbf{wooden box }with 1kg of weight inside, and recorded
			force measured by \textbf{force sensor}

		\item Using data from first trial, recorded \emph{peak static friction }and
			\emph{average kinetic friction}.

		\item Repeated steps 8-9 two more times.

		\item Removed 200 g from \textbf{wooden box}.

		\item Repeated steps 8-11 until 200 g left in \textbf{wooden box}.
	\end{enumerate}


\subsection{Part 3: Kinetic Friction}

\begin{enumerate}[resume]
		\item Removed all mass from \textbf{wooden box.}

		\item Placed \textbf{motion detector }1 - 2 m way from \textbf{wooden box}.

		\item Practiced sliding \textbf{wooden box }towards \textbf{motion detector.}

		\item Gave \textbf{wooden box }a push so it slid towards the \textbf{motion
			detector}.

		\item Recorded acceleration of \textbf{wooden box }using \textbf{Logger Pro}.

		\item Repeated steps 16-17 four more times.

		\item Added 500 g of mass to \textbf{wooden box.}

		\item Repeated steps 16-17 with added mass to \textbf{wooden box.}
	\end{enumerate}

\newpage




