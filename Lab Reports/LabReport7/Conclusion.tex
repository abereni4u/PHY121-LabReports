
\section{Conclusion}

\vspace{-0.5cm}
\singlespacing

The purpose of this lab was to observe static and kinetic friction acting on a wooden block. By measuring the force needed to initally pull and then maintain the block's speed, we were able to determine the static and kinetic friction acting on the block. In \textit{Part 2: Peak Static and Kinetic Friction}, we had a low standard deviation across all our trials, meaning our measurements were precise (see Table~\ref{tab:part2s} and Table~\ref{tab:part2k}). As expected we found that the static friction force was higher than the kinetic friction force (see Table~\ref{tab:part2s} and Table~\ref{tab:part2k}). The relationship between the two is further illustrated in Figure~\ref{fig:SFvNF} and Figure~\ref{fig:KFvNF} where the static coefficient of friction (slope) is higher than the kinetic coefficient of friction. Our results also showed how weight affects the force of friction as the averages consistently lowered as we removed mass from the wooden block (see Table~\ref{tab:part2s} and Table~\ref{tab:part2k}). This makes sense as equations \ref{eq:maxS} and \ref{eq:maxK} show that both static and kinetic friction are proportional with the normal force, which is itself equal to an object's weight (see equation \ref{eq:nForce}). \par

In \textit{Part 3: Kinetic Friction}, we looked at the effect of weight on kinetic friction. After sliding the wooden block in front of a motion sensor and measuring its acceleration, then using that data to calculate the kinetic friction force and the coefficient of kinetic friction, we determined the relationship between speed and weight on kinetic friction. From our data we found that dding weight to the wooden block resulted in a 0.76\% difference in the calculated average coefficient of friction (see Table~\ref{tab:part3block} and Table~\ref{tab:part3500g}). While the force of kinetic friction was generally higher with 500 g of mass added to the block (which is expected based off equation \ref{eq:kFriction}, this had no impact on the coefficient of friction. Since we did not change surfaces, we'd expect the force of friction to change very little, which our data shows.
