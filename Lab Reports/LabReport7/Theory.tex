\section{Theory}
\vspace{-0.5cm}
\singlespacing

\textit{Friction} is the opposing force to an object's motion (Newton's Third Law) upon a surface. There are two types of friction: \textit{static friction}, which acts on an object when it is at rest, and \textit{kinetic friction}, which acts upon objects in motion. The force of static friction is equal to the force applied to the object up to a certain point, called the \textit{maximum static friction}. The maximum force of static friction that can act on an object is determined by the following equation:

\begin{equation}
	F_s <= 	\mu_s F_N
	\label{eq:maxS}
\end{equation}

$\mu_s$ is the \textit{coefficient of friction} which changes based off the surface and is independent of the object. Surfaces of greater roughness have higher coefficients of friction, meaning it requires more force to move an object from rest on those surfaces. $F_N$ refers to the \textit{normal force}, the opposing contact force to gravity that is perpendicular to the surface an object rests on. At rest, the normal force is equal to an object's \textit{weight}. Weight is calculated based of an object's mass, and is dependent on the force of gravity. On Earth, weight can be determined using the following equation: 

\begin{equation}
	F_w = mg	
	\label{eq:weight}
\end{equation}

Both static and kinetic friction are proportional to an object's normal force (increasing weight increases the normal force applied to an object), the latter's amount being determined by the following equation:
\begin{equation}
	F_k = \mu_k F_k
	\label{eq:maxK}
\end{equation}

\newpage

