
 \section{Questions}

\vspace{-0.5cm}
\singlespacing

\begin{enumerate}
	\item \textbf{Which is larger?}

	Static friction is larger.	

	\item \textbf{Based on your graph, would you expect the coefficient of static friction be greater than, less than, or the same as the coefficient of kinetic friction? Why do you say so?}

	The coefficient of static friction has to be larger based off the graph b/c static friction's force is larger than kinetic's. A larger coefficient multiplied by an object's normal force would result in a greater value for the force of friction, which the graph displays. Therefore static friction must have a larger coefficient than kinetic friction.	

\item \textbf{Should a line fitted to these data points pass through the origin?}

No, a line fitted to these points should not pass through the origin, as based off the equation for static friction, both the coefficient of friction would have to be zero, or the normal force would have to be zero. Since the experiment was not conducted on a frictionless surface the coefficient of friction can't be zero, and because we did not conduct the experiment in free fall or zero gravity, the normal force can't be zero either. Therefore the graph shouldn't pass through the origin. 

\item \textbf{Should a line fitted to these points pass through the origin? KINETIC}

Kinetic friction has the same variables in its equation as static friction, so no, for the same reasons mentioned in question 3, a line fitted through these data points shouldn't pass through the origin. 

\item \textbf{Does the coefficient of kinetic friction depend on speed? Explain, using your experimental data.}
	
No, the coefficient of kinetic friction is independent of speed as demonstrated by our data. The average for $\mu_k$ was very similar with no mass added to the box, and with 500 grams of mass added to the box.

\item \textbf{Does the force of kinetic friction depend on the weigh of the box? Explain.}

	The force of kinetic friction \textit{does} depend on the weight of the box as normal force applied to the box is equal to the weight of the box. 

\item \textbf{Does the coefficient of kinetic friction depend on the weight of the box?}

No, the coefficient of friction is independent of the weight of the box. This is evident by our data, as there was very little change between the average $\mu_k$ when no mass was in the box, and when 500 kg of mass was in the box.

\item \textbf{Do you expect them (coefficients of friction from part 2 and part 3) to be the same or different? Explain your reasoning.}

	The coefficients of friction from part 2 and part 3 should be the same since the surface didn't change. The coefficient of friction is correlated with the roughness of a surface. Barring fingerprints from repeated trials, the coefficients shouldn't change much at all since we did not change the location of our experiment. 

\end{enumerate}

