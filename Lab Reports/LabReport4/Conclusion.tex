
\section{Conclusion}

\vspace{-0.5cm}
\singlespacing

The purpose of this lab was to measure the acceleration due to gravity by measuring the velocity of of objects in motion at different times. First we measured the velocity of a picket fence in free fall. Though our average value after six trials wasn't exactly 9.80\,$m/s^2$, we were well within 10\% of error, with our highest being only 1.829\% (see Table~\ref{tab:ff-ta}). Our standard deviation for the free fall trials was low at 0.06439\,$m/s^2$, meaning our measurements were precise (see Table~\ref{tab:ff-ta}). Considering our low relative error and standard deviation, the difference between our values and the accepted value for the acceleration due to gravity can be explained by air resistance. Were the experiment to be conducted again, we could try to minimize air resistance in the room, if not by using a vacuum chamber (unlikely), then by reducing the clutter in the room and improving ventilation throughout it. \par
For the second part of our experiment, we measured the acceleration due to gravity of an object, specifically a cart, on an inclined plane. Since gravity only impacts the vertical motion of objects, we first needed to calculate the expected value for g using equation~\ref{eq:1}. We calculated the expected acceleration due to gravity along our track to be $-0.6105\,m/s^2$. Our trials of the cart when released at the top of the track, produced an average acceleration of $-0.5317\,m/s^2$. Though the standard deviation for our trials was very low at $6.455\,\text{x}\,10^{-3}\,m/s^2$, the relative error for all our trials were more than 10\% (see Table~\ref{tab:ip-sftt}). Besides air resistance, taking our low standard deviation into account, the difference between our measurements and our calculations is most likely due to an incorrect measurement for the angle of the track. When measuring the height and distance of the track, we did not take into account the motion sensor at the bottom of the track, which would influence \textit{both} the height and distance measurements. When conducting this experiment again, greater consideration for the offset caused by the motion sensor should be made.\par 
The standard deviation of the cart when starting from the bottom of the track, also had a low standard deviation at $7.171\,\text{x}\,10^{-3}\,m/s^2$ (Table~\ref{tab:ip-sftb}), but a lower relative error than starting from the top, at 7.567\% (Table~\ref{tab:ip-re}). A review of the graphs (see Figure~\ref{fig:cartg1} & Figure~\ref{fig:cartg2}) shows that the motion sensor had many more data points for velocity when the cart started from the bottom compared to when it started from the top. For that reason, it's likely the acceleration measured when starting from the bottom is closer to the actual acceleration of the cart. Similar to starting from the top, the difference between our measurements and the calculated acceleration along the plane is most likely a result of incorrect measurements for the angle of the track. \par

