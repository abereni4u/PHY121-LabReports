\section{Theory}
\vspace{-0.5cm}
\singlespacing

An object can have two kinds of energy: \textit{kinetic energy} and \textit{potential energy}. Kinetic energy or \textit{KE}, refers to energy in motion and can be determined by the following equation:

\begin{equation*}
	\text{KE} = \frac{1}{2}mv^2
\end{equation*}

Potential energy refers to the stored energy of an object that correlates with its position. Within our experiment, potential energy refers to the \textit{gravitational potential energy} of the object. Gravitational potential energy or \textit{GPE}, is based on an object's height and can be determined by the following equation:

\begin{equation*}
	\text{GPE} = mgh	
\end{equation*}

The sum of kinetic and potential energy is called the \textit{total mechanical energy} of an object, which we can represent using the following equation:

\begin{equation}
	E = \Delta{\text{KPE}} + \Delta{\text{GPE}} 
	\label{eq:totalMechE}
\end{equation}

In an ideal scenario, wherein no \textit{non-conservative} forces that dissipate energy into thermal or other forms, such as \textit{friction}, exist, then the total amount of energy in the system remains constant. In other words, the change in energy of a \textit{conservative} system should be zero, which can be represented in the following equations:

\begin{align*}
	\Delta{\text{KPE}} + \Delta{\text{GPE}} = 0 \\ 
	\Delta{\text{KPE}} = -\Delta{\text{GPE}} 
\end{align*}

\newpage
