
\section{Conclusion}

\vspace{-0.5cm}
\singlespacing

The purpose of this lab was to demonstrate that momentum is conserved during a collision. By analyzing the momentum of carts through simulation and physical means, we were able to confirm the \textit{Conservation of Momentum} law (see equation~\ref{eq:consM}).

In Part 1, we simulated the motion of the elastic collision of two carts and calculated the total momentum and kinetic energy before and after collision. In an elastic collision, both momentum and kinetic energy are conserved. Our results show that that is true. Across all our simulations for Part 1, the percent difference between the total momentum and kinetic energy, after the collision was less than 3\% (see table \ref{tab:part1post}). 

An interesting effect occurred in Part 1 when the red and blue carts had equal mass. Under such circumstances the velocities of the two carts swapped. As an example, in Part 1 trial 3 (see tables \ref{tab:part1pre} and \ref{tab:part1post}) the red cart swapped velocities with the blue cart after collision. The same effect also occurred in trial 1 (see tables \ref{tab:part1pre} and \ref{tab:part1post}). 

For Part 2, we simulated the motion of the inelastic collision of two carts. Once again, we calculated the total momentum and kinetic energy before and after collision. During inelastic collisions, while momentum is conserved, kinetic energy is not. Based on our results, we can confirm that statement. We conducted two trials: one where both carts were equal in mass, and one where they were different (see tables \ref{tab:part2pre} and \ref{tab:part2post}). In both trials, the percent difference in momentum before and after collision was 0\% (see \ref{tab:part2post}). However the percent difference in both trials was much more than 10\%, with trial 1 having a 50\% difference in total kinetic energy post collision, and trial 2 having a 75\% difference (see \ref{tab:part2post}). Since these were inelastic collisions,  the results are to be expected. 

In Part 3, we analyzed the difference in momentum physically using a cart on a track. After tying a rubber band to a Force Sensor and a cart, we measured the force and velocity of the cart after release. An analysis of the graph generated from Logger Pro, allowed us to determine both the change in momentum and the impulse of the cart's motion. Using equations \ref{eq:impulse} and \ref{eq:pdiff}, we found the percent difference between the impulse and momentum to be 4.431\% (see \ref{tab:part3}). Since this difference is less than 10\%, we can confirm that impulse is indeed a change in momentum as described in equation \ref{eq:impulse}. 
