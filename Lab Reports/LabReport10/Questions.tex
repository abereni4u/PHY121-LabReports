
 \section{Questions}

\vspace{-0.5cm}
\singlespacing

\begin{enumerate}
	\item \textbf{What do you notice about the velocity of the second car after the collision when both cars are identical and one starts from rest?}

	The velocity of the second car becomes the same velocity as the first car before collision. In an elastic collision, when both objects have the same mass, the moving object transfers their velocity to the stationary object. 

	\item \textbf{What can you say about the momentum of both cars before the collision and the momentum of the cars after the collision?}

	Before the collision, the red cart has momentum since it's moving, while the blue cart has 0 momentum since it's at rest. After the collision, since both carts have the same mass, the red cart has 0 momentum since its velocity gets transferred to the blue cart, which now has momentum.

\item \textbf{What do you notice about the heavy cart after the collision? Does it go faster or slower than the 5 m/s?}

The heavier cart goes slower than 5 m/s after the collision. 

\item \textbf{What do you notice about the lighter cart after the collision? Which way is it traveling after the collision?}

	The lighter cart goes slower in the opposite direction of its motion prior to the collision. 


\item \textbf{Is kinetic energy conserved? Is momentum conserved?}

	Yes, both kinetic energy and momentum were conserved.


\item \textbf{Write a statement as to what you think should happen to reach cart's motion after the collision.}

	I predict that the red cart's velocity will decrease, and it'll go back in the opposite direction.The blue car's velocity will increase and it'll go towards the right.	
	
\item \textbf{Now seeing what you saw, were you right? What can you say about the momentum before the collision and the momentum after the collision?}

I was correct. Momentum was conserved after the collision. 


\item \textbf{Write what you expect to happen to the carts after they collide. Which will be traveling faster\/slower and in what direction?}

I expect the red cart to go slower in the opposite direction, and the blue cart to go faster and towards the right.
	
\item \textbf{How right were you in your hypothesis of what you thought would happen? Explain why the carts behaved the way they did after the collision using terms like inertia, force, and momentum.}

I thought the red cart would be slower, so I was correct in that aspect, but I was incorrect in regards to the direction it'd be travelling. I thought the blue cart would go faster, which was correct, and I was also correct about its direction after the collision. Inertia is the tendency of an object to stay in its current state, and is directly correlated with the mass of an object. The red cart keeps moving forward, because of inertia, and the blue cart, since its mass is smaller, moves along with the red cart when the red cart's force is applied to it.
	

\item \textbf{How did the total momentum compare to the total momentum after the collision? How did the total kinetic energy after the collision compare to what it started with?}

The total momentum after the collision was the same, but the total kinetic energy was cut in half.
	
\item \textbf{How did the total momentum compare to the total momentum after the collision? How did the total kinetic energy after the collision compare to what it started with?}

Once again, the momentum was the same, and the total kinetic energy was reduced by 75 \%.

\item \textbf{What do you think the major differences are between an elastic and inelastic collision?}

The biggest difference between an elastic and inelastic collision is the conservation of kinetic energy. 

\item \textbf{What is the impulse as calculated by the graph?}

The impulse as calculated by the graph was 0.4724 kg m/s.

\item \textbf{What is the change in momentum, $\mathbf{\Delta{p}}$ as calculated from the graph?}

The change in momentum was calculated by the graph to be 0.4933 kg m/s.

\item \textbf{Based on your answers from question 14 and 15, would you say that the impulse-momentum theorem is approximately verified? Why or why not?}

Based on my answers, I'd say that the impulse-momentum theorem is approximately verified since the change in momentum was calculated to be approximately equal to the impulse.

\item \textbf{How might the force vs. time graph be different for a more stiff rubber band?}

A stiffer rubber band would result in the impulse being greater since the tension force would increase.

\end{enumerate}

%\newpage
