
\section{Conclusion}

\vspace{-0.5cm}
\singlespacing

The purpose of this lab was to use static equilibrium to solve for the center of mass of a rigid rod. By using a meter stick as our rigid rod, and balancing it on a knife edge, we were able to experimentally and through calculation, determine not only its center of mass, but also what positions to place masses on it and maintain static equilibrium. First, we experimentally found the meter stick's center of mass to be 0.509 m (see Table~\ref{tab:s1tab}). Then we placed a 60 g mass 20 cm from one end of the meter stick, applying a torque on that side (see Figure~\ref{fig:Scenario1}). Next we experimentally determined the position to place a 100 g mass to the left of the meter stick's pivot such that the net torque on the meter stick was 0 (see Figure~\ref{fig:Scenario1}). Through experimentation, we determined the position of the 100 g mass to balance the meter stick, to be at 0.333 m on the meter stick (see Table~\ref{tab:s1tab}). Since torque decreases the closer an object is to the pivot point, it makes sense that in order to balance out the torque of the 60 g mass on the other side, the 100 g mass was closer to it than the 60 g mass. Using equation ~\ref{eq:COM} and rearranging it to solve for the expected position of the 100 g mass, we found there to be a 0.4186\% difference with our measured value. 

In the next scenario, we had results that differed a bit from our calculations. We added a 20 g mass to the same side as the 60 g mass, and through experimentation, found the position to place the 100 g mass to be at 0.19 m on the meter stick. Using equation ~\ref{eq:COM}, we found the expected position of the 100 g mass to be 0.2322 m, which is an 18.17\% difference from our measured value (see Table~\ref{tab:s2tab}). Considering that the expected value is closer to the pivot, it's possible that the 20 g mass, and its mass holder, were greater than we thought. Since torque increases the further an object is from the pivot (see ~\ref{eq:torque}), the 100 g we measured was applying a greater torque than the expected value (see Table~\ref{tab:s2tab} and equation~\ref{eq:torque}). This suggests that either the 20 g mass and mass holder were more than 20 g, or the positions of them masses were farther away from the pivot than we recorded. Were I to repeat this experiment again, I would measure the weight of the mass holders as well and ensure that each mass is tightly attached to the meter stick to minimize as much potential sliding as possible. 

In scenario 3 we removed all masses from the meter stick and attached only a 50 g mass, 30 cm from one end of the meter stick. This time we did not add a counter balancing mass to the other end, so we needed to change the meter stick's center of mass. By equation~\ref{eq:torque}, the further an object is from the pivot point, the greater torque it applies. The uniform rod then becomes non uniform, and its distribution of mass unequal. Therefore, to reduce the torque of the 50 g mass, we needed to shift the meter stick's center of mass. Through experimentation we, found that shifting the pivot of the meter stick to 0.6 m, 10 cm behind the 50 g mass, balanced the it (see Figure~\ref{fig:Scenario3}). Using equation~\ref{eq:COM}, we found the expected center of mass to be 0.5833 m, which is only a 2.857\%  from our measured center of mass (see Table~\ref{tab:s3tab}).
