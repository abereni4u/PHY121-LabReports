\section{Theory}
\vspace{-0.5cm}
\singlespacing

\textit{Uniform circular motion} is the motion of an object in a circle at a constant velocity. Although the velocity of the object is constant, the direction is constantly changing in the circle, and thus there is acceleration. This specific type of acceleration, wherein the magnitude of velocity is constant but its direction is changing, is called \textit{centripetal acceleration} and is determined with the following equation.

{\centering{
		\begin{equation*}
			a_c = \frac{v^2}{r}	
		\end{equation*}
\vspace{-0.7cm}
\begin{align*}
	\mathbf{a_c} &: \text{Centripetal acceleration} \\
	\mathbf{v} &: \text{Velocity of object} \\
\mathbf{r} &: \text{Radius of circle} 
\end{align*}
}}

Since there is an acceleration, there must be a force, so using Newton's Second Law, the formula for the \textit{centripetal force} is given by:

{\centering{
		\begin{equation*}
			F_c = m\frac{v^2}{r}	
		\end{equation*}
\vspace{-0.7cm}
\begin{align*}
	\mathbf{F_c} &: \text{Centripetal force} \\
	\mathbf{m} &: \text{Mass of object} \\
	\mathbf{v} &: \text{Velocity of object} \\
	\mathbf{r} &: \text{Radius of circle} 
\end{align*}
}}

Velocity is equal to displacement over time. When dealing with circles, the magnitude of displacement will equal the circumference of the circle. Based off that, the formula for velocity of an object on a circular path is given by:


{\centering{
		\begin{equation*}
			v = \frac{2 \pi r}{t}	
		\end{equation*}
\vspace{-0.7cm}
\begin{align*}
	\mathbf{v} &: \text{Velocity of object} \\
	\mathbf{r} &: \text{Radius of object} \\
	\mathbf{t} &: \text{Time for object to complete on revolution on circular path} \\
	\mathbf{r} &: \text{Radius of circle} 
\end{align*}
}}

\newpage
