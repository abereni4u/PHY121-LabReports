
 \section{Questions}

\vspace{-0.5cm}
\singlespacing

\begin{enumerate}
	\item \textbf{Should the 100 grams be placed closer or further away from the pivot point when compared to the 60 grams?}

	The 100 g should be placed closer so that the torque it applies roughly equals the torque being applied by the 60 g on the other side. 	

	\item \textbf{How does the calculated value compare to the experimental value you found? (Percent error)}

		The calculated value is extremely close to the experimental value we found, with an error of only 0.4186\%.

\item \textbf{How does the calculated value compare to the experimental value you found? (Percent error)}

The calculated value is 4 cm off what we found through experimentation. There error is 18.17\%. This is most likely due to not taking into account the mass of the special mass holder used to hold up the 0.02 g in our calculations.

\item \textbf{Where do you need to place the knife edge such that the meter stick would be balanced?}

	We needed to place the knife edge closer to the 50 g so that the meter stick remained balanced.

\item \textbf{Explain in your own words why the meter stick needed to be shifted to find the new balance point. How can the system balance with one mass placed on it?}

Since there's no other mass on the other side of the meter stick to apply the necessary torque to balance out the torque of the 50 g, the meter stick needed to be shifted. By changing the meter stick's center of mass such that it's closer to the 50 g mass, we reduce the torque that it applies, balancing the rod again.

\item \textbf{Describe the procedure that you used to find the center of mass.}

We unhooked the knife edge and kept sliding the meter stick until it stopped tipping over to the right. 

\item \textbf{What can you deduce about the distribution of mass inside the non uniform rod? (Is one side heavier than the other? How does that affect finding the center of mass?)}

There's an uneven distribution of mass inside a non uniform rod. One side IS heavier than the other, so the center of mass needs to be closer to the heavier side, reducing the torque on that end to make the net torque from left to right closer to 0. 

\end{enumerate}


