
 \section{Questions}

\vspace{-0.5cm}
\singlespacing

\begin{enumerate}
	\item \textbf{Calculate the fractional discrepancy between the slope of your graph and the mass of your system as measured on the balance. Report this as $\frac{\Delta{M}}{M}$\%.}

		We had more trouble with the heavier masses.

	\item \textbf{Two possible sources of systematic error in this experiment are (a) giving the glider a head start, that is, starting up higher than x1 or accidentally giving it a push, and (b) if the air is not turned on high enough and there is friction. Discuss how each of these would affect your result (that is, the mass of the system as determined from the graph, and why. Accordingly, which do you think may have had a greater influence on your result?}

	The biggest source of error in the lab was not following directions properly. Instead of swinging the rubber stopper such that the alligator clip remained 1 cm below the tubing, we attached MORE alligator clips and swung the stopper such that the centripetal force exceeded the weight of the stopper below. Our measured velocity ended up being way higher and consistent than it should have been. 

\item \textbf{What is the y-intercept of your graph? Is this what you expected? If not, what did you expect and why? What is the significance of the y-intercept you actually obtained?}

Besides not following directions properly, part 2 was more difficult since it took some effort to swing the stopper with just the right speed such that the alligator clip remained 1 cm below the tubing.

\item \textbf{During the experiment does the glider accelerate at a rate of $9.8 m/s^2$, slightly less, or significantly less along the frictionless track?}

Our own data shows no correlation, but using algebra, we'd expect the centripetal force to increase as the velocity increased. 
\end{enumerate}

\newpage
