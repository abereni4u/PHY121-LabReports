
\section{Conclusion}

\vspace{-0.5cm}
\singlespacing

The purpose of this lab was to test the conservation of mechanical energy. In a system where no non-conservative forces, such as friction, exist, the change in potential energy of an object should equal its change in kinetic energy. To remove the force of friction as much as possible, a glider was placed on an air track. Under these conditions, we attempted to measure the kinetic energy of the glider between two photogates on the track, and compared it with its calculated gravitational potential energy. We also attempted the measure the impact of mass on the conservation of mechanical energy, and increased add 100 grams of mass to the glider with each trial. Our results were interesting.

The time elapsed between the photogates had a low standard deviation across all our trials, meaning our measurements were precise (see Table~\ref{tab:dataTab}). Using equation's ~\ref{eq:finalV}, ~\ref{eq:energyK} and ~\ref{eq:energyP} we measured how much energy was conserved in the system along the track. We found that across all our trials, the percent difference between the gravitational potential energy and the change in kinetic energy was at least 27\% (see Table~\ref{tab:dataTab}). The consistency of these percent differences suggests two things: firstly, changing the mass did not significantly influence the amount of energy conserved, and secondly, there was some external force, factor, or systemic error that contributed to the difference in energy. 

If we look at the equations for change in kinetic and potential energy and set them equal to each other, it makes sense that mass has no influence on the conservation of energy since it gets cancelled out (see equation's ~\ref{eq:energyK} and ~\ref{eq:energyP}). Velocity and height are the remaining variables in the equations that could impact the conservation of energy. If we assume our measurements for the gravitational potential energy are correct, then it's highly possible the source of the energy differences lies within the measured kinetic energy. The kinetic energy of the glider that was measured and calculated, consistently had an absolute value that was higher than the calculated gravitational potential energy (see Table~\ref{tab:dataTab}). Based off our analysis of equations ~\ref{eq:energyK} and ~\ref{eq:energyP}, we can rule out mass and focus on velocity. A greater velocity results in a greater kinetic energy.Giving the glider a head start on the air track, or incorrectly measuring he distance between the photogates for equation ~\ref{eq:finalV} could have resulted in greater velocity and influenced our calculations for kinetic energy. 

If assume that our measurements for kinetic energy are correct, that the distances between the photogates were accurately measured, and that the glider was not given a head start on the air track, then, ruling out mass, the source of the energy differences would lie in the height measured for the gravitational potential energy calculation. Gravitational potential energy is directly correlated with the height of an object, so a lower height would result in a lower potential energy. Under the assumption that the kinetic energy measurements are correct, it's possible the heights measured for photogate 1 and 2 are lower than they should be, resulting in a lower gravitational potential energy.

Were I to conduct this experiment again, there are a few things I would change to minimize the difference in energies. In order to prevent the possibility of giving the glider a head start, some sort of locking mechanism would need to be added to the air track such that the glider does not need to be let go by human hands. Thereby the initial velocity of the glider is set almost exactly to zero, and the air track is given enough time to negate friction. Regarding measurements, a locking mechanism could also be applied to the photogates, not only to make the time between gates even more consistent, but also to make the starting points for our measurements easier. There was difficulty determining exactly where to start and stop the measurements for the heights of the photogates because of the slanted angle of the air track, so having a locking mechanism would reduce the ambiguity of the measurements. 

