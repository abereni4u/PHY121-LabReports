\section{Theory}
\vspace{-0.5cm}
\singlespacing

Springs when stretched or compressed a certain distance tend to return to their original state. The force required for a spring to return to its original state is called the \textit{restoring force}. The force required for a spring to return to its original position is directly proportional to the displacement of the spring. A greater stretch and\/or compression results in a greater restoring force. The restoring force also changes based on the stiffness of a spring, also known as the \textit{spring constant}, \textit{k}. This relationship between a spring's stiffness, displacement, and restoring force is referred to as \textit{Hooke's Law}, and is represented by the following equation:

\begin{equation}
	\mathbf{F} = -kx
	\label{eq:hookesLaw}
\end{equation}

If you were to hang a mass to a frictionless, massless spring, the time it takes for the spring to oscillate between its original and new positions is called the \textit{period}.

The period of a spring with a hanging mass attached to it can be determined with the following equation:


\begin{equation*}
	\text{T} = 2\pi\sqrt{\frac{m}{k}} \\ 
\end{equation*}

\vspace{-0.5cm}

\begin{align*}
	\text{T} &: \text{period} \\
	m &: \text{mass of object} \\
	k &: \text{spring constant}
\end{align*}

A pendulum, where all mass is located at the bottom, experiences a similar periodic motion to that of a spring. The period of a pendulum can be determined with the following equation:

\begin{equation*}
	\text{T} = 2\pi\sqrt{\frac{l}{g}} \\ 
\end{equation*}

\vspace{-0.5cm}

\begin{align*}
	\text{T} &: \text{period} \\
	l &: \text{length of pendulum} \\
	g &: \text{acceleration due to gravity}
\end{align*}

\newpage
