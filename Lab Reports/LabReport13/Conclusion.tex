
\section{Conclusion}

\vspace{-0.5cm}
\singlespacing

The purpose of this lab was to study how mass distribution affects the period of a spring and a pendulum. To do so we first needed to determine the spring's stiffness, also known as its spring constant, \textit{k}. After placing masses of increasing weight on the spring, and measuring its displacement, we were able to determine the spring's stiffness or k. From our trials and data analysis, we determined the spring's constant to be 8.0051 (see Figure~\ref{fig:p1Graph}). Using this spring constant and equation~\ref{eq:springMass}, we were able to determine the spring's theoretical period if masses were hung from it. Once again, we placed masses of increasing weight on the spring, this time measuring oscillation with a motion sensor. We found that as the weight on the spring increased, so too did its measured period (see Table~\ref{tab:p2tab}). We then calculated the period of an ideal spring with the same weight hung from it. We found that there was some discrepancy between our measured and predicted values. The mass with the greatest fractional discrepancy was 15 grams at 0.0375 , and for the least, 35 grams, at 0.0226 (see Table~\ref{tab:p2tab}). The difference between the measured and predicted values is most likely because the spring is NOT mass less compared to an ideal spring represented in Hooke's Law (see Table~\ref{tab:p1tab}). Were we to re conduct the experiment with a lighter spring, we would most likely see a lower discrepancy between our measured and predicted values.


In part 3, we measured the period of a pendulum under two different conditions: one in which there was a mass attached to the bottom, & one in which there was not. The body of the pendulum we used was a meter stick. Using equation~\ref{eq:pendulum}, we determined the ideal period. Our actual measurements were interesting. We found that the period of the meter stick with no mass attached was farther away from the ideal period than our measurements of the meter stick with a mass attached (see Table~\ref{tab:p3tab}). The reason for this discrepancy is most likely due to the fact that the pendulum referred to in equation~\ref{eq:pendulum} is one in which all its mass is at the bottom. Attaching the mass to the meter stick allowed it to become a pendulum that met that condition. When we removed the mass from the meter stick, it become a uniform rod, with its mass distributed evenly throughout. This would would explain why our measurements of the pendulum's period with a mass attached is closer to the ideal period than the pendulum without a mass attached. 

