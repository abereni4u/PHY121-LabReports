
\section{Conclusion}

\vspace{-0.5cm}
\singlespacing

The purpose of this lab was to explore the relationship between centripetal force and different variables like mass, velocity, and radius that contribute to centripetal force. To do so, we tied a rubber stopper to a string, and the other end of the string through a tube, to a hanging mass with a weight placed on it. By clipping an alligator clip 1 cm below the tube, and then spinning the rubber stopper at such a speed that the alligator clip stayed in place, we could explore the relationships between centripetal force and different variables.

Unfortunately, we did not initially conduct the experiment as described above. Instead of swinging the rubber stopper at just the right speed sch that the alligator clip would remain 1 cm below the tubing, we swuing it much faster, increasing the centripetal force more than necessary, and completely negating the weight of the hanging mass. The result of this error can be seen in our data. Taking a look at Table~\ref{tab:part1Tab} and Figure~\ref{fig:p1Actual}, our values for the velocity of the rubber stopper were incredibly consistent for every 10 g of weight we added to the hanging mass. This should not have been the case: instead of our velocity, and in turn, the centripetal force, increasing alongside the weight of the hanging mass, it stayed mostly the same through out (see Table~\ref{tab:part1Tab} and Figure~\ref{fig:p1Actual}). Because the weight of the hanging mass increased, to keep the alligator clip 1 cm below the tubing, balancing the centripetal force and weight of the hanging mass, we would need to increase the velocity of the rubber stopper by swinging it faster, as Figure~\ref{fig:p1Expected} shows. Table~\ref{tab:part2Tab} shows what the actual velocity of the rubber stopper should have been for the centripetal force to equal the weight of the hanging mass.Our own results for velocity were off by as much as 30\% (see Table~\ref{tab:part2Tab}). This experiment would need to be recreated again, with the directions followed more closely, to get data that better aligns with expectations.

In part 3 we observed the relationship between radius, velocity, and centripetal force. Since we were increasing the radius between each trial, based off equation~\ref{eq:centriForce} we'd expect the velocity to increase along with the radius to equal the weight of the hanging mass. While we managed to follow instructions more closely, we did not sufficiently practice our swings, so our values are still quite far from expectations. For example, while overall the velocity increased between trials, when the radius was 0.15 m, we ended up with a velocity of more than 35\% difference with expected values (see Table~\ref{tab:part3Tab} and Figure~\ref{fig:p3Actual}). Once again, we'd need to repeat this experiment to get values that align more with expectations. Since the weight of the hanging mass did not change between trials, we'd expect the centripetal force to be about the same throughout. Since velocity is directly linked to centripetal force, the percent difference between our actual and expected values for centripetal force are all more than 10\%, reaching as high as 60\% (see Table~\ref{tab:part3Tab}). 

We were not able to directly confirm out hypotheses based on the results of our experiments, due to incorrectly following instructions. However, using equations~\ref{eq:tvelocity} and ~\ref{eq:centriForce}, we determined expected values (see Figure~\ref{fig:p1Expected} and Figure~\ref{fig:p3Expected}). Considering our results, there are a few things I would do differently besides just following directions properly.

The biggest source of error in this lab was undoubtedly human error. We incorrectly followed instructions leading to results that deviated a great deal from expectations. To prevent this mistake from happening in the future, we need to confirm with the lab instructor early on that we're conducting the experiment correctly. We interpreted the instructions in the lab differently than intended, so confirming with the lab instructor our interpretation would have prevented us from making such a mistake. Additionally, we could have checked in with other lab groups to see if their data followed similar trends as ours. For instance, in part 1 of the lab, had we checked with other groups and seen that all their velocities were increasing, we would know immediately that we were doing something incorrect.  
