
 \section{Questions}

\vspace{-0.5cm}
\singlespacing

\begin{enumerate}
	\item \textbf{Bearing in mind that Hooke's Law is F = -kx, determine (and record k from this graph.}

The spring constant of our spring, was 8.0051 based on our graph.	

	\item \textbf{How do your measured results compare to the predicted value of T in each case? For which mass is the fractional discrepancy the greatest? The least? Why do the values differ between the measured and predicted?}
 
	Our measured results are somewhat close to the predicted value for T in each case. The mass with the greatest fractional discrepancy was 15 grams, and for the least, 35 grams. The difference between the measured and predicted values is most likely because the spring is NOT mass less compared to an ideal spring represented in Hooke's Law.  

\item \textbf{There will still be some discrepancy between the predicted and the measured periods. To what might you attribute this?}

The difference between the measured and predicted values is most likely because the spring is NOT mass less compared to an ideal spring represented in Hooke's Law. 

\item \textbf{Which pendulum seemed to come closest to the expected? (The one with the mass at the end or without?) What do you think this meansin regards to the mass of the ruler?}

The one with mass was closer to the expected. This is most likely because the mass of the meter stick is evenly distributed and not all at the bottom, which the equation requires.

\item \textbf{Does the magnitude of the swing matter in determining the period? (Does a bigger amplitude increase the period of the pendulum?) Does Equation 2 have a variable that affects the period?}

No, the magnitude of the swing does not matter in determining the period.

\item \textbf{Were your results consistent with other lab groups? What problems did you experience in this lab that affected your results?}

	Our results were mostly consistent with other groups.

\end{enumerate}

\newpage
