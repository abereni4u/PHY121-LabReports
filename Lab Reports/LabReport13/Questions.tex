
 \section{Questions}

\vspace{-0.5cm}
\singlespacing

\begin{enumerate}
	\item \textbf{Should the 100 grams be placed closer or further away from the pivot point when compared to the 60 grams?}

		We had more trouble with the lighter masses. 

	\item \textbf{How does the calculated value compare to the experimental value you found? (Percent error)}
 
	The biggest source of error in the lab was not following directions properly. Instead of swinging the rubber stopper such that the alligator clip remained 1 cm below the tubing, we attached MORE alligator clips and swung the stopper such that the centripetal force exceeded the weight of the stopper below. Our measured velocity ended up being way higher and consistent than it should have been. 

\item \textbf{How does the calculated value compare to the experimental value you found? (Percent error)}

Since we didn't follow directions properly in part 1, part 3 ended up being the most difficult. I think this was the case because as the radius increased, we needed to find the right speed to balance the weight of the hanging mass at the bottom. We got close but the results would have aligned with our expectations more if we practiced doing the experiment properly the first time.   

\item \textbf{Where do you need to place the knife edge such that the meter stick would be balanced?}

Our own data shows no correlation, but using algebra, we'd expect the centripetal force to increase as the velocity increased. 

\item \textbf{Explain in your own words why the meter stick needed to be shifted to find the new balance point. How can the system balance with one mass placed on it?}

\item \textbf{Describe the procedure that you used to find the center of mass.}

\item \textbf{What can you deduce about the distribution of mass inside the non uniform rod? (Is one side heavier than the other? How does that affect finding the center of mass?)}

\end{enumerate}

\newpage
