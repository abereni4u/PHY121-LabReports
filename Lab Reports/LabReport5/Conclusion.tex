
\section{Conclusion}

\vspace{-0.5cm}
\singlespacing

The purpose of this lab was to use the measurements from a ball launched horizontally to determine the distance that ball would travel when launched at an angle. By applying the equations of kinematics and recognizing the independence of a motion's horizontal and vertical components, we were able to make our calculations. First we measured the average distance of the ball, when launched horizontally five times. We measured an average distance of 1.231 meters (see Table~\ref{tab:HLt}). With a standard deviation of $7.2127\text{x}10^{-2}$ meters (Table~\ref{tab:HLt}), our measurements were precise. We measured the height of the gun to be 0.276 meters (Table~\ref{tab:HLt}) and, by applying equation ~\ref{eq:YMotion}, determined the ball's time in the air to be 0.2373 seconds (Table~\ref{tab:HLt}). Using equation ~\ref{eq:XMotion}, we determined the initial velocity of the ball to be $5.187\,m/s$ (Table~\ref{tab:HLt}).\par
Our calculations for the distance the ball would travel when launched at an incline of $40^\circ$, required us to split the initial velocity into horizontal and vertical components. Since we did not change the compression of the gun, only the angle, we used the initial velocity for x that we calculated in Table~\ref{tab:HLt}. Applying equations ~\ref{eq:YMotion} and ~\ref{eq:XMotion}, we calculated the horizontal distance the ball would travel to be 2.704 meters (Table~\ref{tab:IL40c}). The average distance of our inclined launch trials was  2.902 meters with a standard deviation of 0.1034 meters so our measurements were precise (see Table~\ref{tab:IL40t}). While most our trials were within 10\% error of our calculations, trials 2 and 4 had relative errors of 12.92\% and 10.81\% error respectively (Table~\ref{tab:IL40t}). The higher relative error observed in the data can be attributed to the less than ideal conditions in which the experiment was conducted. External environmental factors such as vibrations of the table on which the gun was placed caused by the presence and activity of other people in the room, could have impacted the launch. To reduce this impact as much as possible, a repeat of this experiment could have the gun on an isolated platform, such that vibrations or similar environmental disturbances cannot affect the gun launch. \par
In contrast to the inclined launch at $40^\circ$, our calculations and subsequent trials for the inclined launch at $55^\circ$ were much closer. We calculated the horizontal distance the gun would travel to be 2.580 meters (Table~\ref{tab:IL55c}). The average distance of our trials at $55^\circ$ was 2.653 meters with a standard deviation of 0.1001 meters (Table~\ref{tab:IL55t}). All of our measurements were within 10\% error, so not only were our measurements precise, they were also accurate. These trials were conducted near the end of class time when less people were around, so it's possible there was less environmental interference on our launches. 
